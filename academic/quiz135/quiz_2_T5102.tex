\documentclass[10pt]{article}

\usepackage{fullpage,graphicx}

\usepackage{geometry}
 \geometry{textheight = 26 cm}
\usepackage[T1]{fontenc}
\usepackage{mathrsfs,amssymb,amsthm,amsmath,verbatim}
\usepackage[usenames]{color}
\usepackage{t1enc}
\usepackage{enumerate}
\usepackage{multicol}
\setlength{\columnsep}{2 cm}
\newcommand{\ds}[1]{\displaystyle{#1}}
\pagenumbering{gobble}

\newcommand{\diff}{\frac{d}{dx}}


%opening
\title{MAT135H1F -- Quiz 2}
\author{TUT5102}
\date{May 28 , 2015}


\newtheorem{quiz}{Question}

\newcommand{\real}{\mathbb{R}}

\begin{document}
\maketitle

FAMILY NAME: ............................................. \hspace{1 cm} GIVEN NAME: ............................... \\



Mark your lecture and tutorial sections: \hspace{1 cm}  STUDENT ID: ............................................. 
\begin{center}


\begin{tabular}{*{6}{|c}|} \hline
L0101 & L5101 & T0101 & T0201 & T5101 & T5102 \\ \hline
& & & & & \\  \hline
\end{tabular}
\end{center}

You have 25 minutes to solve the problems below! Each problem is worth 1 point. Good luck!

\begin{comment}
You have 10 minutes to record your solutions in the boxes below! Good luck!

\begin{center}
\begin{tabular}{*{5}{|c}|} \hline
 Question 1. &  Question 2. & Question 3. &  Question 4. & Question 5.  \\ \hline
&&&& \\ \hline
\end{tabular}
\end{center}
\end{comment}


\vspace{0.5 cm}

% 2 review and 2 easy limit problems, 1 squeeze.
 \begin{multicols}{2}

\begin{quiz} What is the 3rd derivative of $\cos(x)$?
\end{quiz}
\begin {enumerate}[(a)]
\item $\sin(x)$
\item $-\sin(x)$
\item $\cos(x)$
\item $-\cos(x)$
\end{enumerate}
\vspace{0.5 cm}

Answer: (a) $\sin(x)$. We can simply compute the first, second and third derivatives. We have $\frac{d}{dx}\cos(x) = -\sin(x)$ so $\frac{d^2}{dx^2}\cos(x) = -\frac{d}{dx}\sin(x) = -\cos(x)$. Then $\frac{d^3}{dx^3}\cos(x) = -\frac{d}{dx}\cos(x) = \sin(x)$. 

\vspace{5mm}

\begin{quiz} Let $f(x) = e^x \sin(x)$. What is $f'(\frac{ \pi}{2})$?
\end{quiz}
\begin {enumerate}[(a)]
\item $\frac{ \pi}{2} e^{\frac{ \pi}{2}}$
\item $\frac{ \pi}{2}$
\item $\frac{2+ \pi}{2} e^\frac{ \pi}{2}$
\item $e^\frac{ \pi}{2}$
\end{enumerate}
\vspace{0.5 cm}

Answer: (d) $e^{\frac{\pi}{2}}$. First we compute the derivative of $f$ with respect to $x$ by using the product rule. We obtain $f'(x) = e^{x}\sin(x) + e^{x}\cos(x) = e^{x}(\sin(x) + \cos(x))$. Then $f'(\frac{\pi}{2}) = e^{\frac{\pi}{2}}(\sin(\frac{\pi}{2}) + \cos(\frac{\pi}{2})) = e^{\frac{\pi}{2}}$.

\vfill
\columnbreak

\begin{quiz}Let $f(x) = \frac{1}{x} + 3 x$. The equation of the tangent line to the graph of $f(x)$ at $x = \frac{1}{\sqrt{3}}$ is:
\end{quiz}
\begin {enumerate}[(a)]
\item $y= -x + 2\sqrt{3}$
\item $y = -x - 2\sqrt{3}$
\item $y = 0$
\item $y = 2\sqrt{3}$
\end{enumerate}

\vspace{5mm}

Answer: (d) $y = 2\sqrt{3}$. We start by computing $f'(\frac{1}{\sqrt{3}})$ which is the slope of the desired tangent line. We have $f'(x) = \frac{-1}{x^2} + 3$ so $f'(\frac{1}{\sqrt{3}}) = 0$. Thus the tangent line is horizontal with equation $y = b$ for some constant $b$. We know $(\frac{1}{\sqrt{3}}, f(\frac{1}{\sqrt{3}}))$ is a point on the line so we conclude $y = f(\frac{1}{\sqrt{3}}) = 2\sqrt{3}$.

\vspace{5mm}


\begin{quiz} What is the 9th derivative of $1+x+x^2+\dots+x^8$?
\end{quiz}
\begin {enumerate}[(a)]
\item $8\times 7 \times 6 \times 5 \times 4 \times 3 \times 2 \times 1 = 40320$
\item $1$
\item $0$
\item None of the above.
\end{enumerate}
\vspace{0.5 cm}

Answer: (c) 0. We know that the derivative of a sum is equal to the sum of derivatives. This means that we may consider each term of $1 + x + \cdots + x^8$ independently. Differentiating each monomial term once reduces the power of the monomial by one. Thus, differentiating $n$ times a monomial $x^n$ reduces the power by $n$ and leaves us with a constant. Then, differentiating once more gives 0. Namely, differentiating a monomial more times than the degree gives 0. For example $\frac{d^8}{dx^8}x^8 = 8!$ so $\frac{d^9}{dx^9}x^8 = 0$. Therefore, we conclude that $\frac{d^9}{dx^9}(1 + x + \cdots + x^8) = 0$.  

\vspace{5mm}

\begin{quiz} What is $\lim\limits_{h\to 0} \frac{e^{h} - 1}{h}$?
\end{quiz}
\begin {enumerate}[(a)]
\item $-1$
\item $0$
\item $1$
\item $\infty$
\end{enumerate}

\vspace{0.5 cm}

Answer: (c) 1. Let $f(x) = e^x$. We know $f'(a) = f(a)$ for any $a \in \bf{R}$. By definition, $f'(a) =  \lim\limits_{h\to 0} \frac{e^{a+h} - e^a}{h}$. Taking $a=0$, we obtain $f'(0) = \lim\limits_{h\to 0} \frac{e^{h} - 1}{h}$ and $f'(0) = f(0) = 1$.

\vspace{5mm}

\begin{quiz} Sketch the graph of an everywhere continuous function that is not differentiable at $x = 0$ nor at $x=3$.
\end{quiz}
\vspace{0.5 cm}

Answer: A sharp point on a graph is an example of an instance where the left and right limits exist and and equal, the function is defined but there is no derivative. We can draw a graph with a sharp point at $x=0$ and $x=3$.

\includegraphics[width=0.5\textwidth]{blank}

\setlength{\unitlength}{2cm}
\begin{picture}(1,1)
\put(0.05,3){\line(1,-1){1.85}}
\put(1.9,1.15){\line(1,1){1.09}}
\put(2.99,2.25){\line(1,-1){0.8}}
\end{picture}


\end{multicols}



\end{document}


\begin{quiz}
\end{quiz}
\begin {enumerate}[(a)]
\item
\item
\item
\item
\end{enumerate}