\documentclass[10pt]{article}

\usepackage{fullpage,graphicx}

\usepackage{geometry}
 \geometry{textheight = 26 cm}
\usepackage[T1]{fontenc}
\usepackage{mathrsfs,amssymb,amsthm,amsmath,verbatim}
\usepackage[usenames]{color}
\usepackage{t1enc}
\usepackage{enumerate}
\usepackage{multicol}
\setlength{\columnsep}{2 cm}
\newcommand{\ds}[1]{\displaystyle{#1}}
\pagenumbering{gobble}

\newcommand{\diff}{\frac{d}{dx}}


%opening
\title{MAT135H1F -- Quiz 2}
\author{TUT0201}
\date{May 28 , 2015}


\newtheorem{quiz}{Question}

\newcommand{\real}{\mathbb{R}}

\begin{document}
\maketitle

FAMILY NAME: ............................................. \hspace{1 cm} GIVEN NAME: ............................... \\



Mark your lecture and tutorial sections: \hspace{1 cm}  STUDENT ID: ............................................. 
\begin{center}


\begin{tabular}{*{6}{|c}|} \hline
L0101 & L5101 & T0101 & T0201 & T5101 & T5102 \\ \hline
& & & & & \\  \hline
\end{tabular}
\end{center}

You have 25 minutes to solve the problems below! Each problem is worth 1 point. Good luck!

\begin{comment}
You have 10 minutes to record your solutions in the boxes below! Good luck!

\begin{center}
\begin{tabular}{*{5}{|c}|} \hline
 Question 1. &  Question 2. & Question 3. &  Question 4. & Question 5.  \\ \hline
&&&& \\ \hline
\end{tabular}
\end{center}
\end{comment}


\vspace{0.5 cm}

% 2 review and 2 easy limit problems, 1 squeeze.
 \begin{multicols}{2}

\begin{quiz} What is $\lim\limits_{x\to 0} \frac{\cos(x)-1}{x}$?
\end{quiz}
\begin {enumerate}[(a)]
\item $1$
\item $\frac{1}{2}$
\item $0$
\item Does not exist.
\end{enumerate}
\vspace{0.5 cm}

Answer: (c) 0. This limit is standard. It is computed in detail on page 193.

\vspace{5mm}

\begin{quiz} Let $f(x) = \sin^2 (x)$. What is $f'(\frac{ \pi}{4})$?
\end{quiz}
\begin {enumerate}[(a)]
\item $1$
\item $-1$
\item $\frac{1}{\sqrt{2}}$
\item $0$
\end{enumerate}
\vspace{0.5 cm}

Answer: (a) $1$. We differentiate using the chain rule to obtain $\frac{d}{dx}\sin^2(x) = 2\sin(x)\cos(x)$. Thus, evaluating at $\frac{\pi}{4}$ gives $2\sin(\frac{\pi}{4})\cos(\frac{\pi}{4}) = 1$.

\vfill
\columnbreak

\begin{quiz}Let $f(x) =  e^x - e^2 x$. The equation of the tangent line to the graph of $f(x)$ at $x = 2$ is:
\end{quiz}
\begin {enumerate}[(a)]
\item $y= e^2 x + 3$
\item $y = -x + 3(e^2-2)$
\item $y = 0$
\item $y = -e^2$
\end{enumerate}

\vspace{5mm}

Answer: (d) $y = -e^2$. We start by computing $f'(2)$ which is the slope of the desired tangent line. We have $f'(x) = e^x - e^2$ so $f'(2) = 0$. Thus the tangent line is horizontal with equation $y = b$ for some constant $b$. We know $(2, f(2))$ is a point on the line so we conclude $y = f(2) =e^2 - 2e^2 = -e^2$.

\vspace{5mm}

\begin{quiz} What is the 8th derivative of $x^7+x^6+1$?
\end{quiz}
\begin {enumerate}[(a)]
\item $7 \times 6 \times 5 \times 4 \times 3 \times 2 \times 1 = 5040$
\item $1$
\item $0$
\item None of the above.
\end{enumerate}
\vspace{0.5 cm}

Answer: (c) 0. We know that the derivative of a sum is equal to the sum of derivatives. This means that we may consider each term of $x^7 + x^6 + 1$ independently. Differentiating each monomial term once reduces the power of the monomial by one. Thus, differentiating $n$ times a monomial $x^n$ reduces the power by $n$ and leaves us with a constant. Then, differentiating once more gives 0. Namely, differentiating a monomial more times than the degree gives 0. For example $\frac{d^7}{dx^7}x^7 = 7!$ so $\frac{d^8}{dx^8}x^7 = 0$. Therefore, we conclude that $\frac{d^8}{dx^8}(x^7 + x^6 + 1) = 0$.  

\vspace{5mm}

\begin{quiz} What is $\frac{d}{dx}\frac{x}{\cos(x)}$ at $x=0$?
\end{quiz}
\begin {enumerate}[(a)]
\item $0$
\item $1$
\item $-1$
\item Does not exists.
\end{enumerate}

\vspace{0.5 cm}

Answer: (b) 1. Differentiating using the quotient rule gives $\frac{d}{dx}\frac{x}{\cos(x)} = \frac{\cos(x) + x\sin(x)}{\cos^2(x)}$. Thus, evaluating at $x=0$ gives $\frac{\cos(0)}{\cos^2(0)} = 1$.

\vspace{5mm}

\begin{quiz} Sketch the graph of an everywhere continuous function that is not differentiable at $x = -1$ nor at $x = +1$.
\end{quiz}
\vspace{0.5 cm}

Answer: A sharp point on a graph is an example of an instance where the left and right limits exist and and equal, the function is defined but there is no derivative. We can draw a graph with a sharp point at $x=-1$ and $x=1$.

\includegraphics[width=0.5\textwidth]{blank}

\setlength{\unitlength}{2cm}
\begin{picture}(1,1)
\put(0.05,3){\line(2,-1){1.48}}
\put(1.53,2.25){\line(1,2){0.727}}
\put(2.26,3.7){\line(1,-1){1.5}}
\end{picture}

\end{multicols}



\end{document}


\begin{quiz}
\end{quiz}
\begin {enumerate}[(a)]
\item
\item
\item
\item
\end{enumerate}