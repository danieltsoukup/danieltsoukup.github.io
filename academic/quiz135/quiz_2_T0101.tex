\documentclass[10pt]{article}

\usepackage{fullpage,graphicx}

\usepackage{geometry}
 \geometry{textheight = 26 cm}
\usepackage[T1]{fontenc}
\usepackage{mathrsfs,amssymb,amsthm,amsmath,verbatim}
\usepackage[usenames]{color}
\usepackage{t1enc}
\usepackage{enumerate}
\usepackage{multicol}
\setlength{\columnsep}{2 cm}
\newcommand{\ds}[1]{\displaystyle{#1}}
\pagenumbering{gobble}

\newcommand{\diff}{\frac{d}{dx}}


%opening
\title{MAT135H1F -- Quiz 2}
\author{TUT0101}
\date{May 28 , 2015}


\newtheorem{quiz}{Question}

\newcommand{\real}{\mathbb{R}}

\begin{document}
\maketitle

FAMILY NAME: ............................................. \hspace{1 cm} GIVEN NAME: ............................... \\



Mark your lecture and tutorial sections: \hspace{1 cm}  STUDENT ID: ............................................. 
\begin{center}


\begin{tabular}{*{6}{|c}|} \hline
L0101 & L5101 & T0101 & T0201 & T5101 & T5102 \\ \hline
& & & & & \\  \hline
\end{tabular}
\end{center}

You have 25 minutes to solve the problems below! Each problem is worth 1 point. Good luck!

\begin{comment}
You have 10 minutes to record your solutions in the boxes below! 

\begin{center}
\begin{tabular}{*{5}{|c}|} \hline
 Question 1. &  Question 2. & Question 3. &  Question 4. & Question 5.  \\ \hline
&&&& \\ \hline
\end{tabular}
\end{center}
\end{comment}


\vspace{0.5 cm}

% 2 review and 2 easy limit problems, 1 squeeze.
 \begin{multicols}{2}

\begin{quiz} What is $\lim\limits_{x\to 0} \frac{\sin (x)}{2x}$?
\end{quiz}
\begin {enumerate}[(a)]
\item $\frac{1}{2}$
\item 0
\item Does not exist.
\item $\pi$
\end{enumerate}
\vspace{0.5 cm}

Answer: (a) $\frac{1}{2}$. Pulling out a factor of $\frac{1}{2}$ gives $\lim\limits_{x\to 0} \frac{\sin (x)}{2x} = \frac{1}{2}\lim\limits_{x\to 0} \frac{\sin (x)}{x} = \frac{1}{2}$.

\vspace{5mm}

\begin{quiz} Let $f(x) = \frac{e^x}{x^3}$. What is $f'(2)$?
\end{quiz}
\begin {enumerate}[(a)]
\item $-\frac{e^2}{16}$
\item $0$
\item $\frac{e^2}{8}$
\item $-\frac{ \ln2}{16}$
\end{enumerate}
\vspace{0.5 cm}

Answer: (a) $-\frac{e^2}{16}$. Differentiating $f(x)$ with the quotient rule gives $f'(x) =\frac{x^3e^x - 3x^2e^x}{x^6} = \frac{e^x(x - 3)}{x^4}$ so $f'(2) = -\frac{e^2}{16}$.

\vfill
\columnbreak

\begin{quiz}Let $f(x) = \frac{1}{2} e^x - 3x$. The equation of the tangent line to the graph of $f(x)$ at $x = \ln 6$ is:
\end{quiz}
\begin {enumerate}[(a)]
\item $y= e^6 x + 3$
\item $y = -x + 3$
\item $y = 0$
\item $y = 3(1-\ln 6)$
\end{enumerate}

\vspace{5mm}

Answer: (d) $y = 3(1 - \ln6)$. We start by computing $f'(\ln6)$ which is the slope of the desired tangent line. We have $f'(x) = \frac{1}{2}e^x - 3$ so $f'(\ln6) = 0$. Thus the tangent line is horizontal with equation $y = b$ for some constant $b$. We know $(\ln6, f(\ln6))$ is a point on the line so we conclude $y = f(\ln6) = \frac{1}{2}e^{\ln6} - 3\ln6 = 3(1 - \ln6)$.

\vspace{5mm}

\begin{quiz} What is the 8th derivative of $x+x^6-x^7$?
\end{quiz}
\begin {enumerate}[(a)]
\item $7 \times 6 \times 5 \times 4 \times 3 \times 2 \times 1= 5040$
\item $1$
\item $0$
\item None of the above.
\end{enumerate}
\vspace{0.5 cm}

Answer: (c) 0. We know that the derivative of a sum is equal to the sum of derivatives. This means that we may consider each term of $x + x^6 - x^7$ independently. Differentiating each monomial term once reduces the power of the monomial by one. Thus, differentiating $n$ times a monomial $x^n$ reduces the power by $n$ and leaves us with a constant. Then, differentiating once more gives 0. Namely, differentiating a monomial more times than the degree gives 0. For example $\frac{d^7}{dx^7}x^7 = 7!$ so $\frac{d^8}{dx^8}x^7 = 0$. Therefore, we conclude that $\frac{d^8}{dx^8}(x + x^6 - x^7) = 0$.  

\vspace{5mm}

\begin{quiz} What is $\frac{d}{dx} \cos^2(x)$ at $x=\pi/4$?
\end{quiz}
\begin {enumerate}[(a)]
\item $0$
\item $1$
\item $-1$
\item $\frac{1}{\sqrt{2}}$
\end{enumerate}

\vspace{0.5 cm}

Answer: (c) $-1$. We differentiate using the chain rule to obtain $\frac{d}{dx}\cos^2(x) = 2\cos(x)(-\sin(x)) = -2\cos(x)\sin(x)$. Thus, evaluating at $\frac{\pi}{4}$ gives $-2\cos(\frac{\pi}{4})\sin(\frac{\pi}{4}) = -1$.

\vspace{5mm}

\begin{quiz} Sketch the graph of an everywhere continuous function that is not differentiable at $x = 2$.
\end{quiz}
\vspace{0.5 cm}

Answer: A sharp point on a graph is an example of an instance where the left and right limits exist and and equal, the function is defined but there is no derivative. We can draw a graph with a sharp point at $x=2$.

\includegraphics[width=0.5\textwidth]{blank}

\setlength{\unitlength}{2cm}
\begin{picture}(1,1)
\put(0.05,3){\line(3,-1){2.577}}
\put(2.63,2.15){\line(1,1){1.09}}

\end{picture}

\end{multicols}



\end{document}


\begin{quiz}
\end{quiz}
\begin {enumerate}[(a)]
\item
\item
\item
\item
\end{enumerate}